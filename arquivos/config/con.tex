
\usepackage{lmodern}	       % Usa a fonte Latin Modern
\usepackage[table]{xcolor} 
\usepackage[utf8]{inputenc} % Suporte a UTF-8
\usepackage[T1]{fontenc}    % Fontes com suporte a caracteres especiais
\usepackage[portuguese]{babel} % Configuração para português
\usepackage{caption} % Para adicionar legendas
\usepackage[backend=biber, style=abnt]{biblatex} % Estilo ABNT
\usepackage{minted}
\usepackage{pdfpages}
\usepackage{multirow} % Adicione este pacote
\usepackage{chngcntr} % Para alterar a numeração dos contadores
\usepackage{changepage} % Para ajustar recuos
\usepackage{csquotes}       % Recomendado pelo biblatex para citações
\usepackage{setspace}       % Controle de espaçamento
\usepackage{ragged2e}       % Adiciona comandos para justificar texto
\usepackage{hyperref}       % Habilita links clicáveis
\usepackage{subcaption}   
\usepackage[font=small]{caption}
\usepackage{lastpage}			% Usado pela Ficha catalográfica
\usepackage{indentfirst}		% Indenta o primeiro parágrafo de cada seção.
\usepackage{color}				% Controle das cores
\usepackage{graphicx}			% Inclusão de gráficos
\usepackage{microtype} 			% para melhorias de justificação
\usepackage{xcolor} % Permite usar cores no texto
\usepackage{graphicx} % Para incluir imagens
\usepackage{float} % Para controle de posição das figuras

% Definir cores em tons pastel
\definecolor{color_200_200_200}{RGB}{200,200,200}
\definecolor{pastelBlue}{RGB}{135, 206, 250} % Azul pastel moderado
\definecolor{pastelOrange}{RGB}{255, 178, 102} % Laranja pastel equilibrado
\definecolor{pastelRed}{RGB}{240, 128, 128} % Vermelho pastel visível
\definecolor{pastelGreen}{RGB}{144, 238, 144} % Verde pastel

\usepackage{pifont}
\usepackage{amsmath}
\usepackage{amssymb,amsfonts,amsthm}
\usepackage{setspace}
\usepackage{pdfpages} 
\usepackage{hyperref}
\usepackage{adjustbox}
\usepackage{titling} % Para personalizar o título
\usepackage{titlesec}
\usepackage{tocloft} % Para controle do índice
\usepackage{geometry}
\usepackage{etoolbox}
\usepackage{xstring}
\usepackage{hyphenat}
\DeclareGraphicsExtensions{.png,.jpg,.pdf}

\titleformat{\chapter}[hang]
  {\normalfont\huge\bfseries}{}{0pt}{} % Define o estilo do título

% Ajustando as margens
\geometry{
  top=3cm,     % Margem superior de 2cm
  bottom=2cm,  % Margem inferior de 2cm
  left=3cm,    % Margem esquerda de 2cm
  right=2cm    % Margem direita de 2cm
}

% Configurações do Hyperref
\hypersetup{
    colorlinks=true,       % Links coloridos
    linkcolor=blue,        % Cor para links internos
    citecolor=blue2,        % Cor para citações
    urlcolor=blue,         % Cor para URLs
    filecolor=magenta,     
    breaklinks=true        % Permite que os links longos sejam quebrados
}
\definecolor{blue}{RGB}{41,5,195}
\definecolor{blue2}{RGB}{41,5,155}
% Ajuste para melhorar a quebra de links
\Urlmuskip=0mu plus 1mu

\addbibresource{bib.bib}    % Arquivo .bib com as referências 

% Justificar a bibliografia
\renewcommand*{\bibfont}{\sloppy\justifying\small\singlespacing}

% O tamanho do parágrafo é dado por:
\setlength{\parindent}{1.5cm}

% Controle do espaçamento entre um parágrafo e outro:
\setlength{\parskip}{1.5cm}  % tente também \onelineskip


% titulo central
\newcommand{\centersection}[1]{
    \begin{center}
        \textbf{\large #1}
    \end{center}
}
\counterwithout{figure}{chapter} % Remove a dependência de capítulos na numeração das figuras
\renewcommand{\listingscaption}{Código}
\renewcommand{\listoflistingscaption}{Lista de Códigos}


% Redefinindo o ambiente quote globalmente
\usepackage{etoolbox}
\AtBeginEnvironment{quote}{%
  \begin{adjustwidth}{4cm}{0cm} % Recuo de 4 cm à esquerda e nenhum ajuste à direita
  \fontsize{10}{12}\selectfont % Fonte com tamanho 10pt e espaçamento de linha 12pt
}

\AtEndEnvironment{quote}{
  \end{adjustwidth} % Fecha o ambiente adjustwidth
}