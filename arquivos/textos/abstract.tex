
\centersection{ABSTRACT} % Cria o título
     
     Data collection for the analysis of music therapy sessions can be conducted in various ways. There is an abundance of qualitative studies that narrate the functioning of sessions observationally, supported by music therapy evaluation scales. There is a lack of publications in Brazil that utilize quantitative methods for data collection and analysis in Music Therapy using digital equipment, such as musical instruments that transmit information via MIDI (Musical Instrument Digital Interface) to electronic processing. This research project aims to identify the potential factors that hinder access to this form of data collection, explore ways to develop more accessible methods for use by music therapists in Brazil, and assess the feasibility of continuing to develop platforms that use MIDI technology for research. The methodology will be divided into two main workstreams. In the first, participants will interact through a digital MIDI instrument that processes all the musical information played, with musical elements in audio. All information will be processed in a patch developed in MAX/MSP for data collection and in a Python script for quantitative data analysis. In the second workstream, interviews will be conducted with music therapists who use technology in their practices, aiming to map the use of technological tools by music therapists and researchers and conduct an initial validity assessment of the material already developed. This will include initial studies of face, content, and construct validity of the developed protocol, such as analyses of correlations between spontaneous time and anxiety, correlations between synchronization error rates and anticipation skills (related to social cognition), among other psychometric analyses. We hope that this experiment will enable us to quantitatively visualize levels of perception among participants and the relationship between various musical stimuli and musical responses, specifically related to rhythmic aspects such as anticipation, delay, synchronization, perception, memory, and understanding of symbolic structures, among others. This project will contribute to the development of new tools for data collection in Music Therapy, also identifying potential factors that hinder access to this form of data collection and exploring ways to develop more accessible methods for use by music therapists in Brazil.
      
      
      
     \textbf{Keywords}: Music Information Retrieval, Music Therapy, Music and Technology, Music Therapy Analysis.




     \pagenumbering{roman}
     \setcounter{page}{4}  % Começa a numeração romana a partir de 1

  \newpage